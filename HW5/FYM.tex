\documentclass[10pt,twocolumn,letterpaper]{article}

\usepackage{cvpr}
\usepackage{times}
\usepackage{epsfig}
\usepackage{graphicx}
\usepackage{mathtools}
\usepackage{amsmath, amsthm, amssymb}
\usepackage{amssymb}
\usepackage{bm}
\usepackage{subfig} 
\usepackage{geometry}%页面设置  
\usepackage{graphics}%图片设置  
\usepackage{caption}%注释设置  
% Include other packages here, before hyperref.

% If you comment hyperref and then uncomment it, you should delete
% egpaper.aux before re-running latex.  (Or just hit 'q' on the first latex
% run, let it finish, and you should be clear).
\usepackage[breaklinks=true,bookmarks=false]{hyperref}

\cvprfinalcopy % *** Uncomment this line for the final submission

\def\cvprPaperID{****} % *** Enter the CVPR Paper ID here
\def\httilde{\mbox{\tt\raisebox{-.5ex}{\symbol{126}}}}

% Pages are numbered in submission mode, and unnumbered in camera-ready
%\ifcvprfinal\pagestyle{empty}\fi
\setcounter{page}{1}
\begin{document}

%%%%%%%%% TITLE
\title{Report: Beyond Bags of Features: Spatial Pyramid Matching for Recognizing Natural Scene Categories}

\author{Daquan Lin 85610653\\
ShanghaiTech University\\
Shanghai, China\\
{\tt\small lindq@shanghaitech.edu.cn}
% For a paper whose authors are all at the same institution,
% omit the following lines up until the closing ``}''.
% Additional authors and addresses can be added with ``\and'',
% just like the second author.
% To save space, use either the email address or home page, not both
}

\maketitle
%\thispagestyle{empty}

%%%%%%%%% ABSTRACT
\begin{abstract}
   This report introduces how to implement the algorithm for Recognizing Natural Scene Categories by Spatial Pyramid Matching show in this paper\cite{lazebnik2006beyond}. I test algorithm on the publicly available Caltech-256\cite{griffinHolubPerona}.

\end{abstract}

%%%%%%%%% BODY TEXT
\section{Introduction}

Although the Bags of Features has great performance, it is weak in spatial information. People can construct some strange images that consist of some of features without caring about the spatial structure, which can also be classified as a targeted label with high probability. In this paper, the author proposed a new method that contain the spatial information, which called Spatial Pyramid Matching(SPM).

% \par In section 2, I will show how to extract the two datasets and some pretreatment. In section 3, I will discuss how to implement K-means model and some problem I faced and how to fix them. Then, I will compare the performances of two datasets. In section 4, I will talk about how to implement simple Mean Shift. Last, come with a conclusion.

%-------------------------------------------------------------------------
\section{Dataset}
I use a batch of data from \textbf{Caltech-256}, choose $20$ classes and each class pick up $50$ images. Therefore, I get $1000$ images. Then, I split it into training set and test set with ratio of $4/1$.
\section{SIFT and Build Codebook}
For each RBG image, get gray image firstly and calculate the key points for $step size: 4$. Then, calculate the descriptors by key points. For all images, I got $X_train_feature$.\\
\par Set $K = 100$, use K-means to build codebook with $X_train_feature$.
\section{Build Spatial Pyramid}
Construct a sequence of grids at resolutions $l=0,1,2$, such that the grid at level $l$ has $2^l$ cells along each dimension, for a total of $D = 2^{dl}$ cells.
\section{Spatial Pyramid Matching}
For each level, concatenate the pyramid, and calculate the histogram with a weight.
\begin{equation}
\begin{aligned}
\mathcal{K}^L(X, Y) &=\mathcal{I}^L + {\displaystyle \sum_{l=0}^{L-1}\frac{1}{2^{L-l}}(\mathcal{I}^l-\mathcal{I}^{l+1})}\\
&=\frac{1}{2^L}\mathcal{I}^0+{\displaystyle \sum_{l=1}^{L}\frac{1}{2^{L-l+1}}\mathcal{I}^l}
\end{aligned}
\end{equation}
\section{Result}
I use SVM to train the data, When $M = 100, level = 1$, use grid search for $C$ and $\gamma$ by 5-fold Cross-validation. When $C = 100$ and $\gamma = 0.001$, I got the best model that accuracy is $0.431\pm 0.044$ in test dataset.\\
\par When $M = 100, level = 2$, use grid search for $C$ and $\gamma$ by 5-fold Cross-validation. When $C = 10$ and $\gamma = 0.001$, I got the best model that accuracy is $0.398\pm 0.057$ in test dataset.\\
\par Because it is very slow and cost memory heavily, so I just try a little.
% \begin{figure}[htb]
% \centering
%   \begin{tabular}{@{}ccc@{}}
%     \includegraphics[width=.2\textwidth]{2092.png} &
%     \includegraphics[width=.2\textwidth]{8049.png} &
%     % \includegraphics[width=.2\textwidth]{2092.png} &
%     \includegraphics[width=.2\textwidth]{23025.png}   \\
%     \includegraphics[width=.2\textwidth]{2092eps300.png} &
%     \includegraphics[width=.2\textwidth]{8049eps300.png} &
%     % \includegraphics[width=.23\textwidth]{2092.png} &
%     \includegraphics[width=.2\textwidth]{23025eps300.png}   \\
%     \includegraphics[width=.2\textwidth]{2092eps600.png} &
%     \includegraphics[width=.2\textwidth]{8049eps600.png} &
%     % \includegraphics[width=.23\textwidth]{2092.png} &
%     \includegraphics[width=.2\textwidth]{23025eps600.png}   \\
%     % \multicolumn{3}{c}{\includegraphics[width=.2\textwidth]{2092.png}}
%   \end{tabular}
%   \caption{raw 0:$\varepsilon =0$,raw 1:$\varepsilon =300$, raw 2:$\varepsilon =600$ }
% \end{figure}

\section{Conclusion}
In this report, I implemented the algorithm for Recognizing Natural Scene Categories by Spatial Pyramid Matching. 

{\small
\bibliographystyle{ieee}
\bibliography{reference.bib}
}

\end{document}
